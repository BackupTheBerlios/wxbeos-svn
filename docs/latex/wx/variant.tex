%%%%%%%%%%%%%%%%%%%%%%%%%%%%%%%%%%%%%%%%%%%%%%%%%%%%%%%%%%%%%%%%%%%%%%%%%%%%%%%
%% Name:        variant.tex
%% Purpose:     wxVariant docs
%% Author:      wxWidgets Team
%% Modified by:
%% Created:     01/30/2005
%% RCS-ID:      $Id: variant.tex,v 1.21 2006/10/10 17:46:49 JS Exp $
%% Copyright:   (c) wxWidgets Team
%% License:     wxWindows license
%%%%%%%%%%%%%%%%%%%%%%%%%%%%%%%%%%%%%%%%%%%%%%%%%%%%%%%%%%%%%%%%%%%%%%%%%%%%%%%

\section{\class{wxVariant}}\label{wxvariant}

The {\bf wxVariant} class represents a container for any type.
A variant's value can be changed at run time, possibly to a different type of value.

As standard, wxVariant can store values of type bool, wxChar, double, long, string,
string list, time, date, void pointer, list of strings, and list of variants. 
However, an application can extend wxVariant's capabilities by deriving from the
class \helpref{wxVariantData}{wxvariantdata} and using the wxVariantData form of
the wxVariant constructor or assignment operator to assign this data to a variant.
Actual values for user-defined types will need to be accessed via the wxVariantData
object, unlike the case for basic data types where convenience functions such as
\helpref{GetLong}{wxvariantgetlong} can be used.

Pointers to any \helpref{wxObject}{wxobject} derived class can also easily be stored
in a wxVariant. wxVariant will then use wxWidgets' built-in RTTI system to set the
type name (returned by \helpref{GetType}{wxvariantgettype}) and to perform 
type-safety checks at runtime.

This class is useful for reducing the programming for certain tasks, such as an editor
for different data types, or a remote procedure call protocol.

An optional name member is associated with a wxVariant. This might be used, for example,
in CORBA or OLE automation classes, where named parameters are required.

Note that as of wxWidgets 2.7.1, wxVariant is reference counted. Additionally, the
conveniance macros {\bf DECLARE\_VARIANT\_OBJECT} and {\bf IMPLEMENT\_VARIANT\_OBJECT}
were added so that adding (limited) support for conversion to and from wxVariant
can be very easily implemented without modifiying either wxVariant or the class
to be stored by wxVariant. Since assignement operators cannot be declared outside
the class, the shift left operators are used like this:

\begin{verbatim}
    // in the header file
    DECLARE_VARIANT_OBJECT(MyClass)
    
    // in the implementation file
    IMPLMENT_VARIANT_OBJECT(MyClass)
    
    // in the user code
    wxVariant variant;
    MyClass value;
    variant << value;
    
    // or
    value << variant;
\end{verbatim}

For this to work, MyClass must derive from \helpref{wxObject}{wxobject}, implement
the \helpref{wxWidgets RTTI system}{runtimeclassoverview}
and support the assignment operator and equality operator for itself. Ideally, it
should also be reference counted to make copying operations cheap and fast. This
can be most easily implemented using the reference counting support offered by
\helpref{wxObject}{wxobject} itself. By default, wxWidgets already implements
the shift operator conversion for a few of its drawing related classes:

\begin{verbatim}
IMPLEMENT_VARIANT_OBJECT(wxColour)
IMPLEMENT_VARIANT_OBJECT(wxImage)
IMPLEMENT_VARIANT_OBJECT(wxIcon)
IMPLEMENT_VARIANT_OBJECT(wxBitmap)
\end{verbatim}

\wxheading{Derived from}

\helpref{wxObject}{wxobject}

\wxheading{Include files}

<wx/variant.h>

\wxheading{See also}

\helpref{wxVariantData}{wxvariantdata}

\latexignore{\rtfignore{\wxheading{Members}}}

\membersection{wxVariant::wxVariant}\label{wxvariantctor}

\func{}{wxVariant}{\void}

Default constructor.

\func{}{wxVariant}{\param{const wxVariant\& }{variant}}

Copy constructor.

\func{}{wxVariant}{\param{const wxChar*}{ value}, \param{const wxString\& }{name = ``"}}

\func{}{wxVariant}{\param{const wxString\&}{ value}, \param{const wxString\& }{name = ``"}}

Construction from a string value.

\func{}{wxVariant}{\param{wxChar}{ value}, \param{const wxString\& }{name = ``"}}

Construction from a character value.

\func{}{wxVariant}{\param{long}{ value}, \param{const wxString\& }{name = ``"}}

Construction from an integer value. You may need to cast to (long) to
avoid confusion with other constructors (such as the bool constructor).

\func{}{wxVariant}{\param{bool}{ value}, \param{const wxString\& }{name = ``"}}

Construction from a boolean value.

\func{}{wxVariant}{\param{double}{ value}, \param{const wxString\& }{name = ``"}}

Construction from a double-precision floating point value.

\func{}{wxVariant}{\param{const wxList\&}{ value}, \param{const wxString\& }{name = ``"}}

Construction from a list of wxVariant objects. This constructor
copies {\it value}, the application is still responsible for
deleting {\it value} and its contents.

\func{}{wxVariant}{\param{void*}{ value}, \param{const wxString\& }{name = ``"}}

Construction from a void pointer.

\func{}{wxVariant}{\param{wxObject*}{ value}, \param{const wxString\& }{name = ``"}}

Construction from a wxObject pointer.

\func{}{wxVariant}{\param{wxVariantData*}{ data}, \param{const wxString\& }{name = ``"}}

Construction from user-defined data.  The variant holds onto the {\it data} pointer.

\func{}{wxVariant}{\param{wxDateTime\&}{ val}, \param{const wxString\& }{name = ``"}}

Construction from a \helpref{wxDateTime}{wxdatetime}.

\func{}{wxVariant}{\param{wxArrayString\&}{ val}, \param{const wxString\& }{name = ``"}}

Construction from an array of strings.  This constructor copies {\it value} and its contents.

\func{}{wxVariant}{\param{DATE\_STRUCT*}{ val}, \param{const wxString\& }{name = ``"}}

Construction from a odbc date value.  Represented internally by a \helpref{wxDateTime}{wxdatetime} value.

\func{}{wxVariant}{\param{TIME\_STRUCT*}{ val}, \param{const wxString\& }{name = ``"}}

Construction from a odbc time value.  Represented internally by a \helpref{wxDateTime}{wxdatetime} value.

\func{}{wxVariant}{\param{TIMESTAMP\_STRUCT*}{ val}, \param{const wxString\& }{name = ``"}}

Construction from a odbc timestamp value.  Represented internally by a \helpref{wxDateTime}{wxdatetime} value.

\membersection{wxVariant::\destruct{wxVariant}}\label{wxvariantdtor}

\func{}{\destruct{wxVariant}}{\void}

Destructor.

Note that destructor is protected, so wxVariantData cannot usually
be deleted. Instead, \helpref{DecRef}{wxvariantdatadecref} should be called.


\membersection{wxVariant::Append}\label{wxvariantappend}

\func{void}{Append}{\param{const wxVariant\&}{ value}}

Appends a value to the list.

\membersection{wxVariant::Clear}\label{wxvariantclear}

\func{void}{Clear}{\void}

Makes the variant null by deleting the internal data and
set the name to {\it wxEmptyString}.

\membersection{wxVariant::ClearList}\label{wxvariantclearlist}

\func{void}{ClearList}{\void}

Deletes the contents of the list.


\membersection{wxVariant::Convert}\label{wxvariantconvert}

\constfunc{bool}{Convert}{\param{long*}{ value}}

\constfunc{bool}{Convert}{\param{bool*}{ value}}

\constfunc{bool}{Convert}{\param{double*}{ value}}

\constfunc{bool}{Convert}{\param{wxString*}{ value}}

\constfunc{bool}{Convert}{\param{wxChar*}{ value}}

\constfunc{bool}{Convert}{\param{wxDateTime*}{ value}}

Retrieves and converts the value of this variant to the type that {\it value} is.


\membersection{wxVariant::GetCount}\label{wxvariantgetcount}

\constfunc{size\_t}{GetCount}{\void}

Returns the number of elements in the list.

\membersection{wxVariant::Delete}\label{wxvariantdelete}

\func{bool}{Delete}{\param{size\_t }{item}}

Deletes the zero-based {\it item} from the list.

\membersection{wxVariant::GetArrayString}\label{wxvariantgetarraystring}

\constfunc{wxArrayString}{GetArrayString}{\void}

Returns the string array value.

\membersection{wxVariant::GetBool}\label{wxvariantgetbool}

\constfunc{bool}{GetBool}{\void}

Returns the boolean value.

\membersection{wxVariant::GetChar}\label{wxvariantgetchar}

\constfunc{wxChar}{GetChar}{\void}

Returns the character value.

\membersection{wxVariant::GetData}\label{wxvariantgetdata}

\constfunc{wxVariantData*}{GetData}{\void}

Returns a pointer to the internal variant data. To take ownership
of this data, you must call its \helpref{IncRef}{wxvariantdataincref}
method. When you stop using it, \helpref{DecRef}{wxvariantdatadecref}
must be likewise called.

\membersection{wxVariant::GetDateTime}\label{wxvariantgetdatetime}

\constfunc{wxDateTime}{GetDateTime}{\void}

Returns the date value.

\membersection{wxVariant::GetDouble}\label{wxvariantgetdouble}

\constfunc{double}{GetDouble}{\void}

Returns the floating point value.

\membersection{wxVariant::GetLong}\label{wxvariantgetlong}

\constfunc{long}{GetLong}{\void}

Returns the integer value.

\membersection{wxVariant::GetName}\label{wxvariantgetname}

\constfunc{const wxString\&}{GetName}{\void}

Returns a constant reference to the variant name.

\membersection{wxVariant::GetString}\label{wxvariantgetstring}

\constfunc{wxString}{GetString}{\void}

Gets the string value.

\membersection{wxVariant::GetType}\label{wxvariantgettype}

\constfunc{wxString}{GetType}{\void}

Returns the value type as a string. The built-in types are: bool, char, date, double, list, long, string, stringlist, time, void*.

If the variant is null, the value type returned is the string ``null" (not the empty string).

\membersection{wxVariant::GetVoidPtr}\label{wxvariantgetvoidptr}

\constfunc{void*}{GetVoidPtr}{\void}

Gets the void pointer value.

\membersection{wxVariant::GetWxObjectPtr}\label{wxvariantgetwxobjectptr}

\constfunc{wxObject*}{GetWxObjectPtr}{\void}

Gets the wxObject pointer value.

\membersection{wxVariant::Insert}\label{wxvariantinsert}

\func{void}{Insert}{\param{const wxVariant\&}{ value}}

Inserts a value at the front of the list.

\membersection{wxVariant::IsNull}\label{wxvariantisnull}

\constfunc{bool}{IsNull}{\void}

Returns true if there is no data associated with this variant, false if there is data.

\membersection{wxVariant::IsType}\label{wxvariantistype}

\constfunc{bool}{IsType}{\param{const wxString\&}{ type}}

Returns true if {\it type} matches the type of the variant, false otherwise.

\membersection{wxVariant::IsValueKindOf}\label{wxvariantisvaluekindof}

\constfunc{bool}{IsValueKindOf}{\param{const wxClassInfo* type}{ type}}

Returns true if the data is derived from the class described by {\it type}, false otherwise.

\membersection{wxVariant::MakeNull}\label{wxvariantmakenull}

\func{void}{MakeNull}{\void}

Makes the variant null by deleting the internal data.

\membersection{wxVariant::MakeString}\label{wxvariantmakestring}

\constfunc{wxString}{MakeString}{\void}

Makes a string representation of the variant value (for any type).

\membersection{wxVariant::Member}\label{wxvariantmember}

\constfunc{bool}{Member}{\param{const wxVariant\&}{ value}}

Returns true if {\it value} matches an element in the list.

\membersection{wxVariant::NullList}\label{wxvariantnulllist}

\func{void}{NullList}{\void}

Makes an empty list. This differs from a null variant which has no data; a null list
is of type list, but the number of elements in the list is zero.

\membersection{wxVariant::SetData}\label{wxvariantsetdata}

\func{void}{SetData}{\param{wxVariantData*}{ data}}

Sets the internal variant data, deleting the existing data if there is any.

\membersection{wxVariant::operator $=$}\label{wxvariantassignment}

\func{void}{operator $=$}{\param{const wxVariant\& }{value}}

\func{void}{operator $=$}{\param{wxVariantData* }{value}}

\func{void}{operator $=$}{\param{const wxString\& }{value}}

\func{void}{operator $=$}{\param{const wxChar* }{value}}

\func{void}{operator $=$}{\param{wxChar }{value}}

\func{void}{operator $=$}{\param{const long }{value}}

\func{void}{operator $=$}{\param{const bool }{value}}

\func{void}{operator $=$}{\param{const double }{value}}

\func{void}{operator $=$}{\param{void* }{value}}

\func{void}{operator $=$}{\param{wxObject* }{value}}

\func{void}{operator $=$}{\param{const wxList\& }{value}}

\func{void}{operator $=$}{\param{const wxDateTime\& }{value}}

\func{void}{operator $=$}{\param{const wxArrayString\& }{value}}

\func{void}{operator $=$}{\param{const DATE\_STRUCT* }{value}}

\func{void}{operator $=$}{\param{const TIME\_STRUCT* }{value}}

\func{void}{operator $=$}{\param{const TIMESTAMP\_STRUCT* }{value}}

Assignment operators.

\membersection{wxVariant::operator $==$}\label{wxvarianteq}

\constfunc{bool}{operator $==$}{\param{const wxVariant\& }{value}}

\constfunc{bool}{operator $==$}{\param{const wxString\& }{value}}

\constfunc{bool}{operator $==$}{\param{const wxChar* }{value}}

\constfunc{bool}{operator $==$}{\param{wxChar }{value}}

\constfunc{bool}{operator $==$}{\param{const long }{value}}

\constfunc{bool}{operator $==$}{\param{const bool }{value}}

\constfunc{bool}{operator $==$}{\param{const double }{value}}

\constfunc{bool}{operator $==$}{\param{void* }{value}}

\constfunc{bool}{operator $==$}{\param{wxObject* }{value}}

\constfunc{bool}{operator $==$}{\param{const wxList\& }{value}}

\constfunc{bool}{operator $==$}{\param{const wxArrayString\& }{value}}

\constfunc{bool}{operator $==$}{\param{const wxDateTime\& }{value}}

Equality test operators.

\membersection{wxVariant::operator $!=$}\label{wxvariantneq}

\constfunc{bool}{operator $!=$}{\param{const wxVariant\& }{value}}

\constfunc{bool}{operator $!=$}{\param{const wxString\& }{value}}

\constfunc{bool}{operator $!=$}{\param{const wxChar* }{value}}

\constfunc{bool}{operator $!=$}{\param{wxChar }{value}}

\constfunc{bool}{operator $!=$}{\param{const long }{value}}

\constfunc{bool}{operator $!=$}{\param{const bool }{value}}

\constfunc{bool}{operator $!=$}{\param{const double }{value}}

\constfunc{bool}{operator $!=$}{\param{void* }{value}}

\constfunc{bool}{operator $!=$}{\param{wxObject* }{value}}

\constfunc{bool}{operator $!=$}{\param{const wxList\& }{value}}

\constfunc{bool}{operator $!=$}{\param{const wxArrayString\& }{value}}

\constfunc{bool}{operator $!=$}{\param{const wxDateTime\& }{value}}

Inequality test operators.

\membersection{wxVariant::operator $[]$}\label{wxvariantarray}

\constfunc{wxVariant}{operator $[]$}{\param{size\_t }{idx}}

Returns the value at {\it idx} (zero-based).

\func{wxVariant\&}{operator $[]$}{\param{size\_t }{idx}}

Returns a reference to the value at {\it idx} (zero-based). This can be used
to change the value at this index.

\membersection{wxVariant::operator wxChar}\label{wxvariantchar}

\constfunc{char}{operator wxChar}{\void}

Operator for implicit conversion to a wxChar, using \helpref{wxVariant::GetChar}{wxvariantgetchar}.

\membersection{wxVariant::operator double}\label{wxvariantdouble}

\constfunc{double}{operator double}{\void}

Operator for implicit conversion to a double, using \helpref{wxVariant::GetDouble}{wxvariantgetdouble}.

\constfunc{long}{operator long}{\void}

Operator for implicit conversion to a long, using \helpref{wxVariant::GetLong}{wxvariantgetlong}.

\membersection{wxVariant::operator wxString}\label{wxvariantwxstring}

\constfunc{wxString}{operator wxString}{\void}

Operator for implicit conversion to a string, using \helpref{wxVariant::MakeString}{wxvariantmakestring}.

\membersection{wxVariant::operator void*}\label{wxvariantvoid}

\constfunc{void*}{operator void*}{\void}

Operator for implicit conversion to a pointer to a void, using \helpref{wxVariant::GetVoidPtr}{wxvariantgetvoidptr}.

\membersection{wxVariant::operator wxDateTime}\label{wxvariantdatetime}

\constfunc{void*}{operator wxDateTime}{\void}

Operator for implicit conversion to a pointer to a \helpref{wxDateTime}{wxdatetime}, using \helpref{wxVariant::GetDateTime}{wxvariantgetdatetime}.

\section{\class{wxVariantData}}\label{wxvariantdata}

The {\bf wxVariantData} is used to implement a new type for wxVariant. Derive from wxVariantData,
and override the pure virtual functions.

wxVariantData is reference counted, but you don't normally have to care about this, as
wxVariant manages the count automatically. However, incase your application needs to take
ownership of wxVariantData, be aware that the object is created with reference count of 1,
and passing it to wxVariant will not increase this. In other words, \helpref{IncRef}{wxvariantdataincref}
needs to be called only if you both take ownership of wxVariantData and pass it to a wxVariant.
Also note that the destructor is protected, so you can never explicitly delete a wxVariantData
instance. Instead, \helpref{DecRef}{wxvariantdatadecref} will delete the object automatically
when the reference count reaches zero.

\wxheading{Derived from}

\helpref{wxObject}{wxobject}

\wxheading{Include files}

<wx/variant.h>

\wxheading{See also}

\helpref{wxVariant}{wxvariant}

\latexignore{\rtfignore{\wxheading{Members}}}

\membersection{wxVariantData::wxVariantData}\label{wxvariantdatactor}

\func{}{wxVariantData}{\void}

Default constructor.

\membersection{wxVariantData::DecRef}\label{wxvariantdatadecref}

\func{void}{DecRef}{\void}

Decreases reference count. If the count reaches zero, the object is
automatically deleted.

Note that destructor of wxVariantData is protected, so delete
cannot be used as normal. Instead, DecRef should be called.

\membersection{wxVariantData::Eq}\label{wxvariantdataeq}

\constfunc{bool}{Eq}{\param{wxVariantData\&}{ data}}

Returns true if this object is equal to {\it data}.

\membersection{wxVariantData::GetType}\label{wxvariantdatagettype}

\constfunc{wxString}{GetType}{\void}

Returns the string type of the data.

\membersection{wxVariantData::GetValueClassInfo}\label{wxvariantdatagetvalueclassinfo}

\constfunc{wxClassInfo*}{GetValueClassInfo}{\void}

If the data is a wxObject returns a pointer to the objects wxClassInfo structure, if
the data isn't a wxObject the method returns NULL.

\membersection{wxVariantData::IncRef}\label{wxvariantdataincref}

\func{void}{IncRef}{\void}

Increases reference count. Note that initially wxVariantData has reference count of 1.

\membersection{wxVariantData::Read}\label{wxvariantdataread}

\func{bool}{Read}{\param{ostream\&}{ stream}}

\func{bool}{Read}{\param{wxString\&}{ string}}

Reads the data from {\it stream} or {\it string}.

\membersection{wxVariantData::Write}\label{wxvariantdatawrite}

\constfunc{bool}{Write}{\param{ostream\&}{ stream}}

\constfunc{bool}{Write}{\param{wxString\&}{ string}}

Writes the data to {\it stream} or {\it string}.


\membersection{wxGetVariantCast}\label{wxgetvariantcast}

\func{classname *}{wxGetVariantCast}{wxVariant\&, classname}

This macro returns the data stored in {\it variant} cast to the type {\it classname *} if
the data is of this type (the check is done during the run-time) or
{\tt NULL} otherwise.


\wxheading{See also}

\helpref{RTTI overview}{runtimeclassoverview}\\
\helpref{wxDynamicCast}{wxdynamiccast}

